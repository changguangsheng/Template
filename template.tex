\documentclass[a4paper,6pt]{article}
\usepackage{xeCJK}
\usepackage{verbatim}
\usepackage{geometry}  % 页边距
\geometry{left=2.0cm,right=2.0cm,top=2.5cm,bottom=2.5cm}
\usepackage{fancyhdr}  % 页眉页脚
\usepackage{minted}    % 代码高亮
\usepackage[colorlinks]{hyperref}  % 目录可跳转
\setlength{\headheight}{15pt}

% 定义页眉页脚
\pagestyle{fancy}
\fancyhf{}
\fancyhead[C]{ACM Template by Parallelc}
\lfoot{}
\cfoot{\thepage}
\rfoot{}

\author{Parallelc}   
\title{ACM Template}
\begin{document} 
\begin{titlepage}
\maketitle
\thispagestyle{empty}
\vspace*{100pt}
\inputminted[breaklines]{c++}{other/vim.txt}
\end{titlepage}

\newpage
\tableofcontents

\newpage
\section{图论}
\subsection{拓扑排序}
\inputminted[breaklines]{c++}{graph-theory/topsort.cpp}
\subsection{最小生成树}
\subsubsection{Prim堆优化}
\inputminted[breaklines]{c++}{graph-theory/MST/Prim+heap.cpp}
\subsubsection{Kruskal}
\inputminted[breaklines]{c++}{graph-theory/MST/Kruskal.cpp}
\subsection{最短路}
\subsubsection{Dijkstra堆优化}
\inputminted[breaklines]{c++}{graph-theory/short-path/Dijkstra+heap.cpp}
\subsubsection{SPFA}
\inputminted[breaklines]{c++}{graph-theory/short-path/spfa.cpp}
\subsubsection{k短路}
\inputminted[breaklines]{c++}{graph-theory/short-path/k-short.cpp}
\subsection{网络流}
\subsubsection{最大流}
\inputminted[breaklines]{c++}{graph-theory/network-flows/dinic.cpp}
\subsubsection{费用流}
\inputminted[breaklines]{c++}{graph-theory/network-flows/minCostMaxflow.cpp}
\subsection{二分图匹配}
\subsubsection{匈牙利算法}
\paragraph{(1) DFS}
\inputminted[breaklines]{c++}{graph-theory/bipartite-graph/hungary-dfs.cpp}
\paragraph{(2) BFS}
\inputminted[breaklines]{c++}{graph-theory/bipartite-graph/hungary-bfs.cpp}
\subsubsection{最大权匹配}
\inputminted[breaklines]{c++}{graph-theory/bipartite-graph/KM.cpp}
\subsection{最近公共祖先}
\subsubsection{ST}
\inputminted[breaklines]{c++}{graph-theory/LCA/ST.cpp}
\subsubsection{倍增}
\inputminted[breaklines]{c++}{graph-theory/LCA/doubly.cpp}
\subsubsection{tarjan}
\inputminted[breaklines]{c++}{graph-theory/LCA/tarjan.cpp}
\subsection{dfs序}
\inputminted[breaklines]{c++}{graph-theory/dfsx.cpp}
\subsection{树链剖分}
\inputminted[breaklines]{c++}{graph-theory/HLD.cpp}
\subsection{连通性}
\subsubsection{割点-桥}
\inputminted[breaklines]{c++}{graph-theory/connect/cut-bridge.cpp}
\subsubsection{点双连通分量}
\inputminted[breaklines]{c++}{graph-theory/connect/point-DCC.cpp}
\subsubsection{边双连通分量}
\inputminted[breaklines]{c++}{graph-theory/connect/edge-DCC.cpp}
\subsubsection{有向图强连通分量}
\inputminted[breaklines]{c++}{graph-theory/connect/DG-tarjan.cpp}
\subsubsection{2sat}
\inputminted[breaklines]{c++}{graph-theory/connect/2sat.cpp}

\newpage
\section{数论}
\subsection{快速幂}
\inputminted[breaklines]{c++}{math/quickpow.cpp}
\subsection{矩阵快速幂}
\inputminted[breaklines]{c++}{math/matrix-quickpow.cpp}
\subsection{组合数取模}
\inputminted[breaklines]{c++}{math/Cmod.cpp}
\subsection{斯特林数}
\inputminted[breaklines]{c++}{math/Stirling.cpp}
\subsection{素数打表}
\inputminted[breaklines]{c++}{math/prime-table.cpp}
\subsection{逆元}
\inputminted[breaklines]{c++}{math/inverse.cpp}
\subsection{逆元打表}
\inputminted[breaklines]{c++}{math/all-inverse.cpp}
\subsection{不定方程}
\inputminted[breaklines]{c++}{math/indeterminate-equation.cpp}
\subsection{中国剩余定理}
\inputminted[breaklines]{c++}{math/CRT.cpp}
\subsection{高斯消元}
\inputminted[breaklines]{c++}{math/Gauss-elim.cpp}
\subsection{容斥原理}
\inputminted[breaklines]{c++}{math/inclusion-exclusion.cpp}
\subsection{1-n异或和}
\inputminted[breaklines]{c++}{math/xor_n.cpp}
\subsection{BM}
\inputminted[breaklines]{c++}{math/Berlekamp-Massey.cpp}
\subsection{线性递推}
\inputminted[breaklines]{c++}{math/Linear_Seq.cpp}
\subsection{FFT}
\inputminted[breaklines]{c++}{math/fft.cpp}
\subsection{NTT}
\inputminted[breaklines]{c++}{math/ntt.cpp}
\subsection{Java大数}
\inputminted[breaklines]{java}{math/Main.java}
\subsection{多项式}
\inputminted[breaklines]{c++}{math/polynomial.cpp}
\subsection{素数个数$n^{\frac{3}{4}}$}
\inputminted[breaklines]{c++}{math/short-prime-num.cpp}
\subsection{素数个数$n^{\frac{2}{3}}$}
\inputminted[breaklines]{c++}{math/prime-num.cpp}
\subsection{等比数列求和}
\inputminted[breaklines]{c++}{math/Geometric-sequence-sum.cpp}
\subsection{自然数幂和}
\inputminted[breaklines]{c++}{math/sum-of-n^k.cpp}

\newpage
\section{数据结构}
\subsection{树状数组}
\inputminted[breaklines]{c++}{data-structure/BIT.cpp}
\subsection{二维树状数组}
\inputminted[breaklines]{c++}{data-structure/2-BIT.cpp}
\subsection{线段树}
\inputminted[breaklines]{c++}{data-structure/segment-tree.cpp}
\subsection{可持久化线段树}
\inputminted[breaklines]{c++}{data-structure/persistence-tree-tag.cpp}
\subsection{主席树}
\inputminted[breaklines]{c++}{data-structure/chairman-tree.cpp}
\subsection{Treap}
\inputminted[breaklines]{c++}{data-structure/Treap.cpp}
\subsection{可持久化并查集}
\inputminted[breaklines]{c++}{data-structure/rope-bcj.cpp}
\subsection{分块}
\inputminted[breaklines]{c++}{data-structure/Part_Blocks.cpp}
\subsection{莫队算法}
\inputminted[breaklines]{c++}{data-structure/Mo.cpp}

\newpage
\section{字符串}
\subsection{KMP}
\inputminted[breaklines]{c++}{string/KMP.cpp}
\subsection{shift-or}
\inputminted[breaklines]{c++}{string/shift-or.cpp}
\subsection{哈希}
\inputminted[breaklines]{c++}{string/hash.cpp}
\subsection{最小表示法}
\inputminted[breaklines]{c++}{string/min-represent.cpp}
\subsection{子序列匹配}
\inputminted[breaklines]{c++}{string/subseq.cpp}

\newpage
\section{动态规划}
\subsection{最长公共子序列}
\inputminted[breaklines]{c++}{dynamic-programming/LCS.cpp}
\subsection{最长上升子序列}
\inputminted[breaklines]{c++}{dynamic-programming/LIS-nlogn.cpp}
\subsection{区间最值}
\inputminted[breaklines]{c++}{dynamic-programming/RMQ.cpp}
\subsection{背包}
\subsubsection{01背包}
\inputminted[breaklines]{c++}{dynamic-programming/bags/01.cpp}
\subsubsection{超大01背包}
\inputminted[breaklines]{c++}{dynamic-programming/bags/big-01.cpp}
\subsubsection{完全背包}
\inputminted[breaklines]{c++}{dynamic-programming/bags/complete.cpp}

\newpage
\section{类}
\subsection{大整数类}
\inputminted[breaklines]{c++}{class/BigInt.cpp}
\subsection{分数类}
\inputminted[breaklines]{c++}{class/Frac.cpp}
\subsection{矩阵类}
\inputminted[breaklines]{c++}{class/Matrix.cpp}

\newpage
\section{计算几何}
\inputminted[breaklines]{c++}{geometry.cpp}

\newpage
\section{搜索}
\subsection{A*}
\inputminted[breaklines]{c++}{search/Astar.cpp}
\subsection{模拟退火}
\inputminted[breaklines]{c++}{search/SA.cpp}

\newpage
\section{其他}
\subsection{读入挂}
\inputminted[breaklines]{c++}{other/fastIO.cpp}
\subsection{离散化}
\inputminted[breaklines]{c++}{other/LSH.cpp}
\subsection{编译优化}
\inputminted[breaklines]{c++}{other/Ofast.cpp}
\subsection{扩展}
\inputminted[breaklines]{c++}{other/ext.cpp}

\end{document}